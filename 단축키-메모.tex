%	-------------------------------------------------------------------------------
%
%		단축키 
%
%		작성
%		2022년
%		08월
%		05일
%		금요일
%		작성
%
%
%
%	-------------------------------------------------------------------------------

%\documentclass[10pt,xcolor=pdftex,dvipsnames,table]{beamer}
%\documentclass[10pt,blue,xcolor=pdftex,dvipsnames,table,handout]{beamer}
%\documentclass[14pt,blue,xcolor=pdftex,dvipsnames,table,handout]{beamer}
\documentclass[aspectratio=1610,17pt,xcolor=pdftex,dvipsnames,table,handout]{beamer}

		% Font Size
		%	default font size : 11 pt
		%	8,9,10,11,12,14,17,20
		%
		% 	put frame titles 
		% 		1) 	slideatop
		%		2) 	slide centered
		%
		%	navigation bar
		% 		1)	compress
		%		2)	uncompressed
		%
		%	Color
		%		1) blue
		%		2) red
		%		3) brown
		%		4) black and white	
		%
		%	Output
		%		1)  	[default]	
		%		2)	[handout]		for PDF handouts
		%		3) 	[trans]		for PDF transparency
		%		4)	[notes=hide/show/only]

		%	Text and Math Font
		% 		1)	[sans]
		% 		2)	[sefif]
		%		3) 	[mathsans]
		%		4)	[mathserif]


		%	---------------------------------------------------------	
		%	슬라이드 크기 설정 ( 128mm X 96mm )
		%	---------------------------------------------------------	
%			\setbeamersize{text margin left=2mm}
%			\setbeamersize{text margin right=2mm}

	%	========================================================== 	Package
		\usepackage{kotex}						% 한글 사용
		\usepackage{amssymb,amsfonts,amsmath}	% 수학 수식 사용
		\usepackage{color}					%
		\usepackage{colortbl}					%


	%		========================================================= 	note 옵션인 
	%			\setbeameroption{show only notes}
		

	%		========================================================= 	Theme

		%	---------------------------------------------------------	
		%	전체 테마
		%	---------------------------------------------------------	
		%	테마 명명의 관례 : 도시 이름
%			\usetheme{default}			%
%			\usetheme{Madrid}    		%
%			\usetheme{CambridgeUS}    	% -red, no navigation bar
%			\usetheme{Antibes}			% -blueish, tree-like navigation bar

		%	----------------- table of contents in sidebar
			\usetheme{Berkeley}		% -blueish, table of contents in sidebar
									% 개인적으로 마음에 듬

%			\usetheme{Marburg}			% - sidebar on the right
%			\usetheme{Hannover}		% 왼쪽에 마크
%			\usetheme{Berlin}			% - navigation bar in the headline
%			\usetheme{Szeged}			% - navigation bar in the headline, horizontal lines
%			\usetheme{Malmoe}			% - section/subsection in the headline

%			\usetheme{Singapore}
%			\usetheme{Amsterdam}

		%	---------------------------------------------------------	
		%	색 테마
		%	---------------------------------------------------------	
%			\usecolortheme{albatross}	% 바탕 파란
%			\usecolortheme{crane}		% 바탕 흰색
%			\usecolortheme{beetle}		% 바탕 회색
%			\usecolortheme{dove}		% 전체적으로 흰색
%			\usecolortheme{fly}		% 전체적으로 회색
%			\usecolortheme{seagull}	% 휜색
%			\usecolortheme{wolverine}	& 제목이 노란색
%			\usecolortheme{beaver}

		%	---------------------------------------------------------	
		%	Inner Color Theme 			내부 색 테마 ( 블록의 색 )
		%	---------------------------------------------------------	

%			\usecolortheme{rose}		% 흰색
%			\usecolortheme{lily}		% 색 안 칠한다
%			\usecolortheme{orchid} 	% 진하게

		%	---------------------------------------------------------	
		%	Outter Color Theme 		외부 색 테마 ( 머리말, 고리말, 사이드바 )
		%	---------------------------------------------------------	

%			\usecolortheme{whale}		% 진하다
%			\usecolortheme{dolphin}	% 중간
%			\usecolortheme{seahorse}	% 연하다

		%	---------------------------------------------------------	
		%	Font Theme 				폰트 테마
		%	---------------------------------------------------------	
%			\usfonttheme{default}		
			\usefonttheme{serif}			
%			\usefonttheme{structurebold}			
%			\usefonttheme{structureitalicserif}			
%			\usefonttheme{structuresmallcapsserif}			



		%	---------------------------------------------------------	
		%	Inner Theme 				
		%	---------------------------------------------------------	

%			\useinnertheme{default}
			\useinnertheme{circles}		% 원문자			
%			\useinnertheme{rectangles}		% 사각문자			
%			\useinnertheme{rounded}			% 깨어짐
%			\useinnertheme{inmargin}			




		%	---------------------------------------------------------	
		%	이동 단추 삭제
		%	---------------------------------------------------------	
%			\setbeamertemplate{navigation symbols}{}

		%	---------------------------------------------------------	
		%	문서 정보 표시 꼬리말 적용
		%	---------------------------------------------------------	
%			\useoutertheme{infolines}


			
	%	---------------------------------------------------------- 	배경이미지 지정
%			\pgfdeclareimage[width=\paperwidth,height=\paperheight]{bgimage}{./fig/Chrysanthemum.jpg}
%			\setbeamertemplate{background canvas}{\pgfuseimage{bgimage}}

		%	---------------------------------------------------------	
		% 	본문 글꼴색 지정
		%	---------------------------------------------------------	
%			\setbeamercolor{normal text}{fg=purple}
%			\setbeamercolor{normal text}{fg=red!80}	% 숫자는 투명도 표시


		%	---------------------------------------------------------	
		%	itemize 모양 설정
		%	---------------------------------------------------------	
%			\setbeamertemplate{items}[ball]
%			\setbeamertemplate{items}[circle]
%			\setbeamertemplate{items}[rectangle]






		\setbeamercovered{dynamic}





		% --------------------------------- 	문서 기본 사항 설정
		\setcounter{secnumdepth}{5} 		% 문단 번호 깊이
		\setcounter{tocdepth}{5} 			% 문단 번호 깊이




% ------------------------------------------------------------------------------
% Begin document (Content goes below)
% ------------------------------------------------------------------------------
	\begin{document}
	

			\title{단축키 }

			\author{김대희}

			\date{	작성 : 2022년 08월 05일 금요일 
					수정 : 2022년 08월 05일 금요일}


	%	==========================================================
	%		개정 이력
	%	----------------------------------------------------------
	%	2020.08.13 첫 작성
	%	----------------------------------------------------------

	%	==========================================================
	%
	%	----------------------------------------------------------
		\begin{frame}[plain]
		\titlepage
		\end{frame}



%		\begin{frame} [plain]{목차}
		\begin{frame} {목차}
		\tableofcontents
		\end{frame}

	%	========================================================== 	개요
	%		Frame
	%	----------------------------------------------------------
		\part{기본}
		\frame{\partpage}


		\begin{frame} [plain]{목차}
		\tableofcontents
		\end{frame}
		

	%	 ----------------------------------------------------------
	%	 Frame
	%	 ----------------------------------------------------------
		\section{원도우}
%		\frame [plain] {\sectionpage}
		

		\begin{frame} [t,plain]
			\begin{block} {원도우}
			\begin{itemize}
				\item 7시 39 출발 초량역
				\item 8시 16분 노포 도착
				\item 8시 19분 버스 탑승
				\item 8시 47분 문화관 도착
			\end{itemize}
			\end{block}
		\end{frame}



	%	========================================================== 	봉사
	%		Frame
	%	----------------------------------------------------------
		\part{봉사 }
		\frame{\partpage}


		\begin{frame} [plain]{목차}
		\tableofcontents
		\end{frame}
		


	%	 ----------------------------------------------------------
	%	 Frame
	%	 ----------------------------------------------------------  노
		\section{노션}
%		\frame [plain] {\sectionpage}
		

		\begin{frame} [t,plain]
			\begin{block} {노션}
			\begin{itemize}
					\item [03] 월 백중기도 발열체크
					\item [07] 금 지장재일 발열체크
					\item [17] 월 백중기도 발열체크
					\item [30] 일 범어사 계곡 환경 정리
					\item [31] 월 백중기도 발열체크

			\end{itemize}
			\end{block}
		\end{frame}


	%	 ----------------------------------------------------------
	%	 Frame
	%	 ---------------------------------------------------------- 스케치업
		\section{스케치업}
%		\frame [plain] {\sectionpage}
		

		\begin{frame} [t,plain]
			\begin{block} {스케치업}
			\begin{itemize}
					\item [05] 토 	백중기도 화향 지장재일 발열체크
					\item [17] 목 	범어사 초하루법회 발열체크
					\item 자비의 일일찻집 행사 
			\end{itemize}
			\end{block}
		\end{frame}


	%	 ----------------------------------------------------------
	%	 Frame
	%	 ---------------------------------------------------------- 매스캐드 
		\section{매스캐드 }
%		\frame [plain] {\sectionpage}
		

		\begin{frame} [t,plain]
			\begin{block} {매스캐드 }
			\begin{itemize}
					\item [04] 일	지장재일 발열체크
					\item 	개산대제 및 팔관회
			\end{itemize}
			\end{block}
		\end{frame}
		

	%	 ----------------------------------------------------------
	%	 Frame
	%	 ----------------------------------------------------------  필모라
		\section{필모라}
%		\frame [plain] {\sectionpage}
		

		\begin{frame} [t,plain]
			\begin{block} {필모라}
			\begin{itemize}
					\item [03] 화 	지장재일 발열체크
					\item [15] 일 	초하루법회 발열체크
			\end{itemize}
			\end{block}
		\end{frame}
		
	%	 ----------------------------------------------------------
	%	 Frame
	%	 ----------------------------------------------------------  파이썬
		\section{파이썬}
%		\frame [plain] {\sectionpage}
		

		\begin{frame} [t,plain]
			\begin{block} {파이썬}
			\begin{itemize}
					\item [02] 수		지장재일 발열체크
					\item [11] 금		동문의 밤 및 회장 이취임식 
					\item [20] 일		동지새알 빚기
					\item [21] 월		동지팥죽나눔행사
					\item [31] 금		범어사 타종식
					\item 범어사 김장
			\end{itemize}
			\end{block}
		\end{frame}


	%	 ----------------------------------------------------------
	%	 Frame
	%	 ----------------------------------------------------------  한글
		\section{한글}
%		\frame [plain] {\sectionpage}
		

		\begin{frame} [t,plain]
			\begin{block} {한글}
			\begin{itemize}
					\item [02] 수		지장재일 발열체크
					\item [11] 금		동문의 밤 및 회장 이취임식 
					\item [20] 일		동지새알 빚기
					\item [21] 월		동지팥죽나눔행사
					\item [31] 금		범어사 타종식
					\item 범어사 김장
			\end{itemize}
			\end{block}
		\end{frame}


	%	========================================================== 	MS 
	%		Frame
	%	----------------------------------------------------------
		\part{MS }
		\frame{\partpage}


		\begin{frame} [plain]{목차}
		\tableofcontents
		\end{frame}
		

		
				
		
	%	 ----------------------------------------------------------
	%	 Frame
	%	 ---------------------------------------------------------- 엑셀
		\section{엑셀}
%		\frame [plain] {\sectionpage}
		

		\begin{frame} [t,plain]
			\begin{block} {엑셀}
			\begin{itemize}
				\item 휴무
				\item 동아대 병원 정기 검진 
			\end{itemize}
			\end{block}
		\end{frame}


	%	 ----------------------------------------------------------
	%	 Frame
	%	 ---------------------------------------------------------- 파워 포인트 
		\section{파워 포인트 }
%		\frame [plain] {\sectionpage}
		

		\begin{frame} [t,plain]
			\begin{block} {파워 포인트 }
			\begin{itemize}
				\item 기수 회장 총무 회의 8월 19일 수 
				\item 오후 7시
				\item 문화관 법당
			\end{itemize}
			\end{block}
		\end{frame}




	%	========================================================== 	그림 
	%		Frame
	%	----------------------------------------------------------
		\part{그림}
		\frame{\partpage}


		\begin{frame} [plain]{목차}
		\tableofcontents
		\end{frame}
		

	%	 ---------------------------------------------------------- 
	%	 Frame
	%	 ---------------------------------------------------------- 픽슬러 
		\section{픽슬러 }
		\frame [plain] {\sectionpage}

		\begin{frame} [t,plain]
			\begin{block} {픽슬러}
			\begin{itemize}
				\item 8월 13 목
  				\item 19학번 주간 (삼보회)
				\item  간편식
			\end{itemize}
			\end{block}
		\end{frame}

	%	 ---------------------------------------------------------- 
	%	 Frame
	%	 ---------------------------------------------------------- 피그마 
		\section{피그마}
		\frame [plain] {\sectionpage}

		\begin{frame} [t,plain]
			\begin{block} {피그마}
			\begin{itemize}
				\item 8월 13 목
  				\item 19학번 주간 (삼보회)
				\item  간편식
			\end{itemize}
			\end{block}
		\end{frame}


	%	========================================================== 	
	%		Frame
	%	---------------------------------------------------------- 어도비 
		\part{어도비}
		\frame{\partpage}


		\begin{frame} [plain]{목차}
		\tableofcontents
		\end{frame}
		


	%	 ---------------------------------------------------------- 
	%	 Frame
	%	 ---------------------------------------------------------- 어도비
		\section{어도비}
		\frame [plain] {\sectionpage}

		\begin{frame} [t,plain]
			\begin{block} {어도비}
			\begin{itemize}
				\item 잘하자 18 ** ( * )
  				\item 21 80 17 23 ( 1 )
				\item ! gy 4615 379 
			\end{itemize}
			매주 수요일 패스워드 변경 
			\end{block}
		\end{frame}




% ------------------------------------------------------------------------------ ------------------------------------------------------------------------------ ------------------------------------------------------------------------------
% End document
% ------------------------------------------------------------------------------ ------------------------------------------------------------------------------ ------------------------------------------------------------------------------
\end{document}

